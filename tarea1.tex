% Options for packages loaded elsewhere
\PassOptionsToPackage{unicode}{hyperref}
\PassOptionsToPackage{hyphens}{url}
%
\documentclass[
]{article}
\usepackage{lmodern}
\usepackage{amssymb,amsmath}
\usepackage{ifxetex,ifluatex}
\ifnum 0\ifxetex 1\fi\ifluatex 1\fi=0 % if pdftex
  \usepackage[T1]{fontenc}
  \usepackage[utf8]{inputenc}
  \usepackage{textcomp} % provide euro and other symbols
\else % if luatex or xetex
  \usepackage{unicode-math}
  \defaultfontfeatures{Scale=MatchLowercase}
  \defaultfontfeatures[\rmfamily]{Ligatures=TeX,Scale=1}
\fi
% Use upquote if available, for straight quotes in verbatim environments
\IfFileExists{upquote.sty}{\usepackage{upquote}}{}
\IfFileExists{microtype.sty}{% use microtype if available
  \usepackage[]{microtype}
  \UseMicrotypeSet[protrusion]{basicmath} % disable protrusion for tt fonts
}{}
\makeatletter
\@ifundefined{KOMAClassName}{% if non-KOMA class
  \IfFileExists{parskip.sty}{%
    \usepackage{parskip}
  }{% else
    \setlength{\parindent}{0pt}
    \setlength{\parskip}{6pt plus 2pt minus 1pt}}
}{% if KOMA class
  \KOMAoptions{parskip=half}}
\makeatother
\usepackage{xcolor}
\IfFileExists{xurl.sty}{\usepackage{xurl}}{} % add URL line breaks if available
\IfFileExists{bookmark.sty}{\usepackage{bookmark}}{\usepackage{hyperref}}
\hypersetup{
  pdftitle={Tarea 1},
  hidelinks,
  pdfcreator={LaTeX via pandoc}}
\urlstyle{same} % disable monospaced font for URLs
\usepackage[margin=1in]{geometry}
\usepackage{color}
\usepackage{fancyvrb}
\newcommand{\VerbBar}{|}
\newcommand{\VERB}{\Verb[commandchars=\\\{\}]}
\DefineVerbatimEnvironment{Highlighting}{Verbatim}{commandchars=\\\{\}}
% Add ',fontsize=\small' for more characters per line
\usepackage{framed}
\definecolor{shadecolor}{RGB}{248,248,248}
\newenvironment{Shaded}{\begin{snugshade}}{\end{snugshade}}
\newcommand{\AlertTok}[1]{\textcolor[rgb]{0.94,0.16,0.16}{#1}}
\newcommand{\AnnotationTok}[1]{\textcolor[rgb]{0.56,0.35,0.01}{\textbf{\textit{#1}}}}
\newcommand{\AttributeTok}[1]{\textcolor[rgb]{0.77,0.63,0.00}{#1}}
\newcommand{\BaseNTok}[1]{\textcolor[rgb]{0.00,0.00,0.81}{#1}}
\newcommand{\BuiltInTok}[1]{#1}
\newcommand{\CharTok}[1]{\textcolor[rgb]{0.31,0.60,0.02}{#1}}
\newcommand{\CommentTok}[1]{\textcolor[rgb]{0.56,0.35,0.01}{\textit{#1}}}
\newcommand{\CommentVarTok}[1]{\textcolor[rgb]{0.56,0.35,0.01}{\textbf{\textit{#1}}}}
\newcommand{\ConstantTok}[1]{\textcolor[rgb]{0.00,0.00,0.00}{#1}}
\newcommand{\ControlFlowTok}[1]{\textcolor[rgb]{0.13,0.29,0.53}{\textbf{#1}}}
\newcommand{\DataTypeTok}[1]{\textcolor[rgb]{0.13,0.29,0.53}{#1}}
\newcommand{\DecValTok}[1]{\textcolor[rgb]{0.00,0.00,0.81}{#1}}
\newcommand{\DocumentationTok}[1]{\textcolor[rgb]{0.56,0.35,0.01}{\textbf{\textit{#1}}}}
\newcommand{\ErrorTok}[1]{\textcolor[rgb]{0.64,0.00,0.00}{\textbf{#1}}}
\newcommand{\ExtensionTok}[1]{#1}
\newcommand{\FloatTok}[1]{\textcolor[rgb]{0.00,0.00,0.81}{#1}}
\newcommand{\FunctionTok}[1]{\textcolor[rgb]{0.00,0.00,0.00}{#1}}
\newcommand{\ImportTok}[1]{#1}
\newcommand{\InformationTok}[1]{\textcolor[rgb]{0.56,0.35,0.01}{\textbf{\textit{#1}}}}
\newcommand{\KeywordTok}[1]{\textcolor[rgb]{0.13,0.29,0.53}{\textbf{#1}}}
\newcommand{\NormalTok}[1]{#1}
\newcommand{\OperatorTok}[1]{\textcolor[rgb]{0.81,0.36,0.00}{\textbf{#1}}}
\newcommand{\OtherTok}[1]{\textcolor[rgb]{0.56,0.35,0.01}{#1}}
\newcommand{\PreprocessorTok}[1]{\textcolor[rgb]{0.56,0.35,0.01}{\textit{#1}}}
\newcommand{\RegionMarkerTok}[1]{#1}
\newcommand{\SpecialCharTok}[1]{\textcolor[rgb]{0.00,0.00,0.00}{#1}}
\newcommand{\SpecialStringTok}[1]{\textcolor[rgb]{0.31,0.60,0.02}{#1}}
\newcommand{\StringTok}[1]{\textcolor[rgb]{0.31,0.60,0.02}{#1}}
\newcommand{\VariableTok}[1]{\textcolor[rgb]{0.00,0.00,0.00}{#1}}
\newcommand{\VerbatimStringTok}[1]{\textcolor[rgb]{0.31,0.60,0.02}{#1}}
\newcommand{\WarningTok}[1]{\textcolor[rgb]{0.56,0.35,0.01}{\textbf{\textit{#1}}}}
\usepackage{graphicx,grffile}
\makeatletter
\def\maxwidth{\ifdim\Gin@nat@width>\linewidth\linewidth\else\Gin@nat@width\fi}
\def\maxheight{\ifdim\Gin@nat@height>\textheight\textheight\else\Gin@nat@height\fi}
\makeatother
% Scale images if necessary, so that they will not overflow the page
% margins by default, and it is still possible to overwrite the defaults
% using explicit options in \includegraphics[width, height, ...]{}
\setkeys{Gin}{width=\maxwidth,height=\maxheight,keepaspectratio}
% Set default figure placement to htbp
\makeatletter
\def\fps@figure{htbp}
\makeatother
\setlength{\emergencystretch}{3em} % prevent overfull lines
\providecommand{\tightlist}{%
  \setlength{\itemsep}{0pt}\setlength{\parskip}{0pt}}
\setcounter{secnumdepth}{-\maxdimen} % remove section numbering

\title{Tarea 1}
\author{}
\date{\vspace{-2.5em}}

\begin{document}
\maketitle

\hypertarget{problema-1}{%
\subsection{Problema 1}\label{problema-1}}

Tenemos la matriz A y B

\begin{Shaded}
\begin{Highlighting}[]
\NormalTok{A =}\StringTok{ }\KeywordTok{matrix}\NormalTok{(}\DataTypeTok{data=}\KeywordTok{c}\NormalTok{(}\DecValTok{7}\NormalTok{, }\DecValTok{5}\NormalTok{, }\DecValTok{3}\NormalTok{, }\DecValTok{2}\NormalTok{, }\DecValTok{1}\NormalTok{, }\DecValTok{8}\NormalTok{), }\DataTypeTok{byrow =} \OtherTok{TRUE}\NormalTok{, }\DataTypeTok{nrow =} \DecValTok{2}\NormalTok{)}
\NormalTok{B =}\StringTok{ }\KeywordTok{matrix}\NormalTok{(}\DataTypeTok{data=}\KeywordTok{c}\NormalTok{(}\DecValTok{11}\NormalTok{, }\DecValTok{-7}\NormalTok{, }\DecValTok{8}\NormalTok{, }\DecValTok{12}\NormalTok{, }\DecValTok{0}\NormalTok{, }\DecValTok{9}\NormalTok{), }\DataTypeTok{byrow =} \OtherTok{TRUE}\NormalTok{, }\DataTypeTok{nrow =} \DecValTok{2}\NormalTok{)}
\end{Highlighting}
\end{Shaded}

\begin{Shaded}
\begin{Highlighting}[]
\NormalTok{A}
\end{Highlighting}
\end{Shaded}

\begin{verbatim}
##      [,1] [,2] [,3]
## [1,]    7    5    3
## [2,]    2    1    8
\end{verbatim}

\begin{Shaded}
\begin{Highlighting}[]
\NormalTok{B}
\end{Highlighting}
\end{Shaded}

\begin{verbatim}
##      [,1] [,2] [,3]
## [1,]   11   -7    8
## [2,]   12    0    9
\end{verbatim}

\begin{enumerate}
\def\labelenumi{\alph{enumi})}
\tightlist
\item
  A transpuesta
\end{enumerate}

\begin{Shaded}
\begin{Highlighting}[]
\KeywordTok{t}\NormalTok{(A)}
\end{Highlighting}
\end{Shaded}

\begin{verbatim}
##      [,1] [,2]
## [1,]    7    2
## [2,]    5    1
## [3,]    3    8
\end{verbatim}

\begin{enumerate}
\def\labelenumi{\alph{enumi})}
\setcounter{enumi}{1}
\tightlist
\item
  A - B
\end{enumerate}

\begin{Shaded}
\begin{Highlighting}[]
\NormalTok{A }\OperatorTok{-}\StringTok{ }\NormalTok{B}
\end{Highlighting}
\end{Shaded}

\begin{verbatim}
##      [,1] [,2] [,3]
## [1,]   -4   12   -5
## [2,]  -10    1   -1
\end{verbatim}

\begin{enumerate}
\def\labelenumi{\alph{enumi})}
\setcounter{enumi}{2}
\tightlist
\item
  AB no se puede por que las dimensiones no concuerdan
\end{enumerate}

\begin{Shaded}
\begin{Highlighting}[]
\CommentTok{#A %*% B}
\end{Highlighting}
\end{Shaded}

d)A'A

\begin{Shaded}
\begin{Highlighting}[]
\KeywordTok{t}\NormalTok{(A)}\OperatorTok\NormalTok{A}
\end{Highlighting}
\end{Shaded}

\begin{verbatim}
##      [,1] [,2] [,3]
## [1,]   53   37   37
## [2,]   37   26   23
## [3,]   37   23   73
\end{verbatim}

e)AA'

\begin{Shaded}
\begin{Highlighting}[]
\NormalTok{A}\OperatorTok\KeywordTok{t}\NormalTok{(A)}
\end{Highlighting}
\end{Shaded}

\begin{verbatim}
##      [,1] [,2]
## [1,]   83   43
## [2,]   43   69
\end{verbatim}

\begin{enumerate}
\def\labelenumi{\alph{enumi})}
\setcounter{enumi}{5}
\tightlist
\item
  A + B
\end{enumerate}

\begin{Shaded}
\begin{Highlighting}[]
\NormalTok{A }\OperatorTok{+}\StringTok{ }\NormalTok{B}
\end{Highlighting}
\end{Shaded}

\begin{verbatim}
##      [,1] [,2] [,3]
## [1,]   18   -2   11
## [2,]   14    1   17
\end{verbatim}

g)A'B

\begin{Shaded}
\begin{Highlighting}[]
\KeywordTok{t}\NormalTok{(A)}\OperatorTok\NormalTok{B}
\end{Highlighting}
\end{Shaded}

\begin{verbatim}
##      [,1] [,2] [,3]
## [1,]  101  -49   74
## [2,]   67  -35   49
## [3,]  129  -21   96
\end{verbatim}

\begin{enumerate}
\def\labelenumi{\alph{enumi})}
\setcounter{enumi}{7}
\tightlist
\item
  AB'
\end{enumerate}

\begin{Shaded}
\begin{Highlighting}[]
\NormalTok{A}\OperatorTok\KeywordTok{t}\NormalTok{(B)}
\end{Highlighting}
\end{Shaded}

\begin{verbatim}
##      [,1] [,2]
## [1,]   66  111
## [2,]   79   96
\end{verbatim}

\begin{enumerate}
\def\labelenumi{\roman{enumi})}
\tightlist
\item
  17.3A
\end{enumerate}

\begin{Shaded}
\begin{Highlighting}[]
\FloatTok{17.3}\OperatorTok{*}\NormalTok{A}
\end{Highlighting}
\end{Shaded}

\begin{verbatim}
##       [,1] [,2]  [,3]
## [1,] 121.1 86.5  51.9
## [2,]  34.6 17.3 138.4
\end{verbatim}

j)(1/19)B

\begin{Shaded}
\begin{Highlighting}[]
\DecValTok{1}\OperatorTok{/}\DecValTok{19}\OperatorTok{*}\NormalTok{B}
\end{Highlighting}
\end{Shaded}

\begin{verbatim}
##           [,1]       [,2]      [,3]
## [1,] 0.5789474 -0.3684211 0.4210526
## [2,] 0.6315789  0.0000000 0.4736842
\end{verbatim}

\hypertarget{problema-6}{%
\section{Problema 6}\label{problema-6}}

Para la matriz de covarianza alrededor de \(x=a\)
\[S(a) = \frac{1}{n}\sum_{i=1}^{n} (x_{i}-a)(x_{i}-a)' \] a) Por
demostrar : \[S(a) = S + (\overline{x} - a )(\overline{x} - a )' \]

Recordemos que el producto exterior \(f(x,y) = xy'\) es una función
bilineal y tenemos que :

\begin{align*}
&(x_{i}-a)(x_{i}-a)' = \\
&(x_{i}-\overline{x}+\overline{x}-a)(x_{i}-\overline{x}+\overline{x}-a)' = \\ &(x_{i}-\overline{x}+\overline{x}-a)(x_{i}-\overline{x})' + (x_{i}-\overline{x}+\overline{x}-a)(\overline{x}-a)' = \\
&(x_{i}-\overline{x})(x_{i}-\overline{x})'+ (\overline{x}-a)(x_{i}-\overline{x})'+ (x_{i}-\overline{x})(\overline{x}-a)'+ (\overline{x}-a)(\overline{x}-a)' = \\
&(x_{i}-\overline{x})(x_{i}-\overline{x})'+ 2(\overline{x}-a)(x_{i}-\overline{x})'+ (\overline{x}-a)(\overline{x}-a)'
\end{align*}

Así \begin{align*}
&S(a) = \frac{1}{n}\sum_{i=1}^{n} (x_{i}-a)(x_{i}-a)' =\\
&\frac{1}{n}\sum_{i=1}^{n}(x_{i}-\overline{x})(x_{i}-\overline{x})'+ \frac{2}{n}\sum_{i=1}^{n}(\overline{x}-a)(x_{i}-\overline{x})'+ \frac{1}{n}\sum_{i=1}^{n}(\overline{x}-a)(\overline{x}-a)' = \\
&S + 2(\overline{x}-a)(\frac{1}{n}\sum_{i=1}^{n}x_{i}-\overline{x})'+(\overline{x}-a)(\overline{x}-a)' = \\
&S + (\overline{x}-a)(\overline{x}-a)'
\end{align*}

\begin{enumerate}
\def\labelenumi{\alph{enumi})}
\setcounter{enumi}{1}
\tightlist
\item
  De a) podemos calcular el determinante de \(S(a)\) recordando que
  \[det(A+ba') = det(A)(1+b'A^{-1}a)\] Así:
\end{enumerate}

\[det(S(a)) = det(S+(\overline{x}-a)(\overline{x}-a)') = det(S)(1+(\overline{x}-a)'S^{-1}(\overline{x}-a))\]

Además, como \(S\) es semidefinida positiva sus eigenvalores son no
negativos y los eigenvalores de \(S^{-1}\) son los recíprocos y
conservan el signo por lo que tambíen es semidefinida positiva y así la
forma cuadrática \((\overline{x}-a)'S^{-1}(\overline{x}-a)\) tiene un
valor mínimo de cero que se alcanza en \(\overline{x} = a\) :

\[min_{a}det(S(a)) = det(S)\]

\end{document}
