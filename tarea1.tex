% Options for packages loaded elsewhere
\PassOptionsToPackage{unicode}{hyperref}
\PassOptionsToPackage{hyphens}{url}
%
\documentclass[
]{article}
\usepackage{lmodern}
\usepackage{amssymb,amsmath}
\usepackage{ifxetex,ifluatex}
\ifnum 0\ifxetex 1\fi\ifluatex 1\fi=0 % if pdftex
  \usepackage[T1]{fontenc}
  \usepackage[utf8]{inputenc}
  \usepackage{textcomp} % provide euro and other symbols
\else % if luatex or xetex
  \usepackage{unicode-math}
  \defaultfontfeatures{Scale=MatchLowercase}
  \defaultfontfeatures[\rmfamily]{Ligatures=TeX,Scale=1}
\fi
% Use upquote if available, for straight quotes in verbatim environments
\IfFileExists{upquote.sty}{\usepackage{upquote}}{}
\IfFileExists{microtype.sty}{% use microtype if available
  \usepackage[]{microtype}
  \UseMicrotypeSet[protrusion]{basicmath} % disable protrusion for tt fonts
}{}
\makeatletter
\@ifundefined{KOMAClassName}{% if non-KOMA class
  \IfFileExists{parskip.sty}{%
    \usepackage{parskip}
  }{% else
    \setlength{\parindent}{0pt}
    \setlength{\parskip}{6pt plus 2pt minus 1pt}}
}{% if KOMA class
  \KOMAoptions{parskip=half}}
\makeatother
\usepackage{xcolor}
\IfFileExists{xurl.sty}{\usepackage{xurl}}{} % add URL line breaks if available
\IfFileExists{bookmark.sty}{\usepackage{bookmark}}{\usepackage{hyperref}}
\hypersetup{
  pdftitle={Tarea 1},
  pdfauthor={Javier Montiel González, Eliza Zenteno y Andrés Cruz Vega},
  hidelinks,
  pdfcreator={LaTeX via pandoc}}
\urlstyle{same} % disable monospaced font for URLs
\usepackage[margin=1in]{geometry}
\usepackage{color}
\usepackage{fancyvrb}
\newcommand{\VerbBar}{|}
\newcommand{\VERB}{\Verb[commandchars=\\\{\}]}
\DefineVerbatimEnvironment{Highlighting}{Verbatim}{commandchars=\\\{\}}
% Add ',fontsize=\small' for more characters per line
\usepackage{framed}
\definecolor{shadecolor}{RGB}{248,248,248}
\newenvironment{Shaded}{\begin{snugshade}}{\end{snugshade}}
\newcommand{\AlertTok}[1]{\textcolor[rgb]{0.94,0.16,0.16}{#1}}
\newcommand{\AnnotationTok}[1]{\textcolor[rgb]{0.56,0.35,0.01}{\textbf{\textit{#1}}}}
\newcommand{\AttributeTok}[1]{\textcolor[rgb]{0.77,0.63,0.00}{#1}}
\newcommand{\BaseNTok}[1]{\textcolor[rgb]{0.00,0.00,0.81}{#1}}
\newcommand{\BuiltInTok}[1]{#1}
\newcommand{\CharTok}[1]{\textcolor[rgb]{0.31,0.60,0.02}{#1}}
\newcommand{\CommentTok}[1]{\textcolor[rgb]{0.56,0.35,0.01}{\textit{#1}}}
\newcommand{\CommentVarTok}[1]{\textcolor[rgb]{0.56,0.35,0.01}{\textbf{\textit{#1}}}}
\newcommand{\ConstantTok}[1]{\textcolor[rgb]{0.00,0.00,0.00}{#1}}
\newcommand{\ControlFlowTok}[1]{\textcolor[rgb]{0.13,0.29,0.53}{\textbf{#1}}}
\newcommand{\DataTypeTok}[1]{\textcolor[rgb]{0.13,0.29,0.53}{#1}}
\newcommand{\DecValTok}[1]{\textcolor[rgb]{0.00,0.00,0.81}{#1}}
\newcommand{\DocumentationTok}[1]{\textcolor[rgb]{0.56,0.35,0.01}{\textbf{\textit{#1}}}}
\newcommand{\ErrorTok}[1]{\textcolor[rgb]{0.64,0.00,0.00}{\textbf{#1}}}
\newcommand{\ExtensionTok}[1]{#1}
\newcommand{\FloatTok}[1]{\textcolor[rgb]{0.00,0.00,0.81}{#1}}
\newcommand{\FunctionTok}[1]{\textcolor[rgb]{0.00,0.00,0.00}{#1}}
\newcommand{\ImportTok}[1]{#1}
\newcommand{\InformationTok}[1]{\textcolor[rgb]{0.56,0.35,0.01}{\textbf{\textit{#1}}}}
\newcommand{\KeywordTok}[1]{\textcolor[rgb]{0.13,0.29,0.53}{\textbf{#1}}}
\newcommand{\NormalTok}[1]{#1}
\newcommand{\OperatorTok}[1]{\textcolor[rgb]{0.81,0.36,0.00}{\textbf{#1}}}
\newcommand{\OtherTok}[1]{\textcolor[rgb]{0.56,0.35,0.01}{#1}}
\newcommand{\PreprocessorTok}[1]{\textcolor[rgb]{0.56,0.35,0.01}{\textit{#1}}}
\newcommand{\RegionMarkerTok}[1]{#1}
\newcommand{\SpecialCharTok}[1]{\textcolor[rgb]{0.00,0.00,0.00}{#1}}
\newcommand{\SpecialStringTok}[1]{\textcolor[rgb]{0.31,0.60,0.02}{#1}}
\newcommand{\StringTok}[1]{\textcolor[rgb]{0.31,0.60,0.02}{#1}}
\newcommand{\VariableTok}[1]{\textcolor[rgb]{0.00,0.00,0.00}{#1}}
\newcommand{\VerbatimStringTok}[1]{\textcolor[rgb]{0.31,0.60,0.02}{#1}}
\newcommand{\WarningTok}[1]{\textcolor[rgb]{0.56,0.35,0.01}{\textbf{\textit{#1}}}}
\usepackage{graphicx,grffile}
\makeatletter
\def\maxwidth{\ifdim\Gin@nat@width>\linewidth\linewidth\else\Gin@nat@width\fi}
\def\maxheight{\ifdim\Gin@nat@height>\textheight\textheight\else\Gin@nat@height\fi}
\makeatother
% Scale images if necessary, so that they will not overflow the page
% margins by default, and it is still possible to overwrite the defaults
% using explicit options in \includegraphics[width, height, ...]{}
\setkeys{Gin}{width=\maxwidth,height=\maxheight,keepaspectratio}
% Set default figure placement to htbp
\makeatletter
\def\fps@figure{htbp}
\makeatother
\setlength{\emergencystretch}{3em} % prevent overfull lines
\providecommand{\tightlist}{%
  \setlength{\itemsep}{0pt}\setlength{\parskip}{0pt}}
\setcounter{secnumdepth}{-\maxdimen} % remove section numbering

\title{Tarea 1}
\author{Javier Montiel González, Eliza Zenteno y Andrés Cruz Vega}
\date{24/8/2020}

\begin{document}
\maketitle

\hypertarget{problema-1}{%
\subsection{Problema 1}\label{problema-1}}

Tenemos la matriz A y B

\begin{Shaded}
\begin{Highlighting}[]
\NormalTok{A =}\StringTok{ }\KeywordTok{matrix}\NormalTok{(}\DataTypeTok{data=}\KeywordTok{c}\NormalTok{(}\DecValTok{7}\NormalTok{, }\DecValTok{5}\NormalTok{, }\DecValTok{3}\NormalTok{, }\DecValTok{2}\NormalTok{, }\DecValTok{1}\NormalTok{, }\DecValTok{8}\NormalTok{), }\DataTypeTok{byrow =} \OtherTok{TRUE}\NormalTok{, }\DataTypeTok{nrow =} \DecValTok{2}\NormalTok{)}
\NormalTok{B =}\StringTok{ }\KeywordTok{matrix}\NormalTok{(}\DataTypeTok{data=}\KeywordTok{c}\NormalTok{(}\DecValTok{11}\NormalTok{, }\DecValTok{-7}\NormalTok{, }\DecValTok{8}\NormalTok{, }\DecValTok{12}\NormalTok{, }\DecValTok{0}\NormalTok{, }\DecValTok{9}\NormalTok{), }\DataTypeTok{byrow =} \OtherTok{TRUE}\NormalTok{, }\DataTypeTok{nrow =} \DecValTok{2}\NormalTok{)}
\end{Highlighting}
\end{Shaded}

\begin{Shaded}
\begin{Highlighting}[]
\NormalTok{A}
\end{Highlighting}
\end{Shaded}

\begin{verbatim}
##      [,1] [,2] [,3]
## [1,]    7    5    3
## [2,]    2    1    8
\end{verbatim}

\begin{Shaded}
\begin{Highlighting}[]
\NormalTok{B}
\end{Highlighting}
\end{Shaded}

\begin{verbatim}
##      [,1] [,2] [,3]
## [1,]   11   -7    8
## [2,]   12    0    9
\end{verbatim}

\begin{enumerate}
\def\labelenumi{\alph{enumi})}
\tightlist
\item
  A transpuesta
\end{enumerate}

\begin{Shaded}
\begin{Highlighting}[]
\KeywordTok{t}\NormalTok{(A)}
\end{Highlighting}
\end{Shaded}

\begin{verbatim}
##      [,1] [,2]
## [1,]    7    2
## [2,]    5    1
## [3,]    3    8
\end{verbatim}

\begin{enumerate}
\def\labelenumi{\alph{enumi})}
\setcounter{enumi}{1}
\tightlist
\item
  A - B
\end{enumerate}

\begin{Shaded}
\begin{Highlighting}[]
\NormalTok{A }\OperatorTok{-}\StringTok{ }\NormalTok{B}
\end{Highlighting}
\end{Shaded}

\begin{verbatim}
##      [,1] [,2] [,3]
## [1,]   -4   12   -5
## [2,]  -10    1   -1
\end{verbatim}

\begin{enumerate}
\def\labelenumi{\alph{enumi})}
\setcounter{enumi}{2}
\tightlist
\item
  AB no se puede por que las dimensiones no concuerdan
\end{enumerate}

\begin{Shaded}
\begin{Highlighting}[]
\CommentTok{#A %*% B}
\end{Highlighting}
\end{Shaded}

d)A'A

\begin{Shaded}
\begin{Highlighting}[]
\KeywordTok{t}\NormalTok{(A)}\OperatorTok\NormalTok{A}
\end{Highlighting}
\end{Shaded}

\begin{verbatim}
##      [,1] [,2] [,3]
## [1,]   53   37   37
## [2,]   37   26   23
## [3,]   37   23   73
\end{verbatim}

e)AA'

\begin{Shaded}
\begin{Highlighting}[]
\NormalTok{A}\OperatorTok\KeywordTok{t}\NormalTok{(A)}
\end{Highlighting}
\end{Shaded}

\begin{verbatim}
##      [,1] [,2]
## [1,]   83   43
## [2,]   43   69
\end{verbatim}

\begin{enumerate}
\def\labelenumi{\alph{enumi})}
\setcounter{enumi}{5}
\tightlist
\item
  A + B
\end{enumerate}

\begin{Shaded}
\begin{Highlighting}[]
\NormalTok{A }\OperatorTok{+}\StringTok{ }\NormalTok{B}
\end{Highlighting}
\end{Shaded}

\begin{verbatim}
##      [,1] [,2] [,3]
## [1,]   18   -2   11
## [2,]   14    1   17
\end{verbatim}

g)A'B

\begin{Shaded}
\begin{Highlighting}[]
\KeywordTok{t}\NormalTok{(A)}\OperatorTok\NormalTok{B}
\end{Highlighting}
\end{Shaded}

\begin{verbatim}
##      [,1] [,2] [,3]
## [1,]  101  -49   74
## [2,]   67  -35   49
## [3,]  129  -21   96
\end{verbatim}

\begin{enumerate}
\def\labelenumi{\alph{enumi})}
\setcounter{enumi}{7}
\tightlist
\item
  AB'
\end{enumerate}

\begin{Shaded}
\begin{Highlighting}[]
\NormalTok{A}\OperatorTok\KeywordTok{t}\NormalTok{(B)}
\end{Highlighting}
\end{Shaded}

\begin{verbatim}
##      [,1] [,2]
## [1,]   66  111
## [2,]   79   96
\end{verbatim}

\begin{enumerate}
\def\labelenumi{\roman{enumi})}
\tightlist
\item
  17.3A
\end{enumerate}

\begin{Shaded}
\begin{Highlighting}[]
\FloatTok{17.3}\OperatorTok{*}\NormalTok{A}
\end{Highlighting}
\end{Shaded}

\begin{verbatim}
##       [,1] [,2]  [,3]
## [1,] 121.1 86.5  51.9
## [2,]  34.6 17.3 138.4
\end{verbatim}

j)(1/19)B

\begin{Shaded}
\begin{Highlighting}[]
\DecValTok{1}\OperatorTok{/}\DecValTok{19}\OperatorTok{*}\NormalTok{B}
\end{Highlighting}
\end{Shaded}

\begin{verbatim}
##           [,1]       [,2]      [,3]
## [1,] 0.5789474 -0.3684211 0.4210526
## [2,] 0.6315789  0.0000000 0.4736842
\end{verbatim}

\#\#Problema 3

Trabajamos con el conjunto de datos \texttt{iris3}.

\begin{enumerate}
\def\labelenumi{\alph{enumi})}
\tightlist
\item
  Calcular \(\boldsymbol{\bar{x}}\), la matriz de sumas de cuadrados
  corregida por la media \(\boldsymbol{A}\) y la matriz de covarianza
  insesgada \(S_X\).
\end{enumerate}

Tomamos la matriz \(X\) con los datos correspondientes a la especie
Setosa. Se obtuvo lo siguiente:

\begin{Shaded}
\begin{Highlighting}[]
\NormalTok{x_barra}
\end{Highlighting}
\end{Shaded}

\begin{verbatim}
## [1] 5.006 3.428 1.462 0.246
\end{verbatim}

\begin{Shaded}
\begin{Highlighting}[]
\NormalTok{A}
\end{Highlighting}
\end{Shaded}

\begin{verbatim}
##                 Sepal L..Setosa Sepal W..Setosa Petal L..Setosa Petal W..Setosa
## Sepal L..Setosa          6.0882          4.8616          0.8014          0.5062
## Sepal W..Setosa          4.8616          7.0408          0.5732          0.4556
## Petal L..Setosa          0.8014          0.5732          1.4778          0.2974
## Petal W..Setosa          0.5062          0.4556          0.2974          0.5442
\end{verbatim}

\begin{Shaded}
\begin{Highlighting}[]
\NormalTok{S_x}
\end{Highlighting}
\end{Shaded}

\begin{verbatim}
##                 Sepal L..Setosa Sepal W..Setosa Petal L..Setosa Petal W..Setosa
## Sepal L..Setosa      0.12424898     0.099216327     0.016355102     0.010330612
## Sepal W..Setosa      0.09921633     0.143689796     0.011697959     0.009297959
## Petal L..Setosa      0.01635510     0.011697959     0.030159184     0.006069388
## Petal W..Setosa      0.01033061     0.009297959     0.006069388     0.011106122
\end{verbatim}

b)Obtener eigenvalores y eigenvectores de \(\boldsymbol{S}\).

\begin{Shaded}
\begin{Highlighting}[]
\NormalTok{(spec <-}\StringTok{ }\KeywordTok{eigen}\NormalTok{(S_x))}
\end{Highlighting}
\end{Shaded}

\begin{verbatim}
## eigen() decomposition
## $values
## [1] 0.236455690 0.036918732 0.026796399 0.009033261
## 
## $vectors
##            [,1]       [,2]       [,3]        [,4]
## [1,] 0.66907840  0.5978840  0.4399628 -0.03607712
## [2,] 0.73414783 -0.6206734 -0.2746075 -0.01955027
## [3,] 0.09654390  0.4900556 -0.8324495 -0.23990129
## [4,] 0.06356359  0.1309379 -0.1950675  0.96992969
\end{verbatim}

c)Sea \(U\) la matrix de eigenvectores de \(S\) y \(L\) la matriz
cuadrada cuyas entradas en la diagonal son los eigenvalores.

Demostramos numéricamente que \(\boldsymbol{ULU'}=\boldsymbol{S}\)
calculando la norma euclideana de la diferencia de matrices y notemos
que dichas cantidades son casi cero.

\begin{Shaded}
\begin{Highlighting}[]
\KeywordTok{norm}\NormalTok{(U}\OperatorTok\NormalTok{L}\OperatorTok\KeywordTok{t}\NormalTok{(U) }\OperatorTok{-}\StringTok{ }\NormalTok{S_x)}
\end{Highlighting}
\end{Shaded}

\begin{verbatim}
## [1] 8.262488e-15
\end{verbatim}

Vemos también que \(\boldsymbol{ULU'}\) y \(\boldsymbol{S}\) son
prácticamente iguales.

\begin{Shaded}
\begin{Highlighting}[]
\NormalTok{U}\OperatorTok\NormalTok{L}\OperatorTok\KeywordTok{t}\NormalTok{(U)}
\end{Highlighting}
\end{Shaded}

\begin{verbatim}
##            [,1]        [,2]        [,3]        [,4]
## [1,] 0.12424898 0.099216327 0.016355102 0.010330612
## [2,] 0.09921633 0.143689796 0.011697959 0.009297959
## [3,] 0.01635510 0.011697959 0.030159184 0.006069388
## [4,] 0.01033061 0.009297959 0.006069388 0.011106122
\end{verbatim}

\begin{Shaded}
\begin{Highlighting}[]
\NormalTok{S_x}
\end{Highlighting}
\end{Shaded}

\begin{verbatim}
##                 Sepal L..Setosa Sepal W..Setosa Petal L..Setosa Petal W..Setosa
## Sepal L..Setosa      0.12424898     0.099216327     0.016355102     0.010330612
## Sepal W..Setosa      0.09921633     0.143689796     0.011697959     0.009297959
## Petal L..Setosa      0.01635510     0.011697959     0.030159184     0.006069388
## Petal W..Setosa      0.01033061     0.009297959     0.006069388     0.011106122
\end{verbatim}

Observamos que
\(\boldsymbol{U'U}=\boldsymbol{UU'}=\boldsymbol{I}_4\).(Se tomó un
redondeo de 3 dígitos)

\begin{Shaded}
\begin{Highlighting}[]
\KeywordTok{round}\NormalTok{(U}\OperatorTok\KeywordTok{t}\NormalTok{(U),}\DecValTok{3}\NormalTok{) }\CommentTok{#UU'}
\end{Highlighting}
\end{Shaded}

\begin{verbatim}
##      [,1] [,2] [,3] [,4]
## [1,]    1    0    0    0
## [2,]    0    1    0    0
## [3,]    0    0    1    0
## [4,]    0    0    0    1
\end{verbatim}

\begin{Shaded}
\begin{Highlighting}[]
\KeywordTok{round}\NormalTok{(}\KeywordTok{t}\NormalTok{(U)}\OperatorTok\NormalTok{U,}\DecValTok{3}\NormalTok{) }\CommentTok{#U'U}
\end{Highlighting}
\end{Shaded}

\begin{verbatim}
##      [,1] [,2] [,3] [,4]
## [1,]    1    0    0    0
## [2,]    0    1    0    0
## [3,]    0    0    1    0
## [4,]    0    0    0    1
\end{verbatim}

Obtener matriz de gráficas de dispersión de las cuatro variables para
cada variedad de iris, todo en la misma gráfica.
\includegraphics{tarea1_files/figure-latex/unnamed-chunk-22-1.pdf}

\#\#Problema 4:

Usar los datos del ejercicio anterior.

a)Considerando
\(Y=\begin{pmatrix}y_1 & y_2 & y_3 & y_4 & y_5\end{pmatrix}\) donde
\(y_i\) representa la \(i\)-ésima columna de \(Y\) cuyas primeras cuatro
columnas sean las mismas que X y cuya última columna es Petal L. + Petal
W. con una matriz \(C\) tal que \(Y=XC\).

Tomamos \(C\) de la forma:
\[C=\begin{pmatrix}1&0&0&0&0\\0&1&0&0&0\\0&0&1&0&1\\0&0&0&1&1\end{pmatrix}\]

\begin{verbatim}
##      [,1] [,2] [,3] [,4] [,5]
## [1,]  5.1  3.5  1.4  0.2  1.6
## [2,]  4.9  3.0  1.4  0.2  1.6
## [3,]  4.7  3.2  1.3  0.2  1.5
## [4,]  4.6  3.1  1.5  0.2  1.7
## [5,]  5.0  3.6  1.4  0.2  1.6
## [6,]  5.4  3.9  1.7  0.4  2.1
\end{verbatim}

b)Calcular la matriz de covarianzas muestral \(S_Y\) y sus eigenvalores
y eigenvectores

\begin{verbatim}
## eigen() decomposition
## $values
## [1] 2.442194e-01 7.483824e-02 3.305869e-02 1.049178e-02 7.456453e-19
## 
## $vectors
##            [,1]        [,2]        [,3]        [,4]          [,5]
## [1,] 0.65694047  0.03846434  0.75289193 -0.01017121  3.463130e-16
## [2,] 0.71184975  0.29542749 -0.63659061 -0.02729311 -2.573590e-16
## [3,] 0.12521671 -0.54795807 -0.08917354 -0.58547201 -5.773503e-01
## [4,] 0.07555878 -0.20605014 -0.04479513  0.78517149 -5.773503e-01
## [5,] 0.20077549 -0.75400821 -0.13396867  0.19969948  5.773503e-01
\end{verbatim}

\begin{Shaded}
\begin{Highlighting}[]
\NormalTok{y_barra}
\end{Highlighting}
\end{Shaded}

\begin{verbatim}
## [1] 5.006 3.428 1.462 0.246 1.708
\end{verbatim}

\begin{Shaded}
\begin{Highlighting}[]
\NormalTok{A_Y}
\end{Highlighting}
\end{Shaded}

\begin{verbatim}
##        [,1]   [,2]   [,3]   [,4]   [,5]
## [1,] 6.0882 4.8616 0.8014 0.5062 1.3076
## [2,] 4.8616 7.0408 0.5732 0.4556 1.0288
## [3,] 0.8014 0.5732 1.4778 0.2974 1.7752
## [4,] 0.5062 0.4556 0.2974 0.5442 0.8416
## [5,] 1.3076 1.0288 1.7752 0.8416 2.6168
\end{verbatim}

\begin{Shaded}
\begin{Highlighting}[]
\NormalTok{S_y}
\end{Highlighting}
\end{Shaded}

\begin{verbatim}
##            [,1]        [,2]        [,3]        [,4]       [,5]
## [1,] 0.12424898 0.099216327 0.016355102 0.010330612 0.02668571
## [2,] 0.09921633 0.143689796 0.011697959 0.009297959 0.02099592
## [3,] 0.01635510 0.011697959 0.030159184 0.006069388 0.03622857
## [4,] 0.01033061 0.009297959 0.006069388 0.011106122 0.01717551
## [5,] 0.02668571 0.020995918 0.036228571 0.017175510 0.05340408
\end{verbatim}

Notemos que el valor más pequeño de los eigenvalores es \emph{cercano} a
cero. De la misma forma, la variable \texttt{comb\_lin} toma la
combinación lineal de las columnas de \(Y\) cuyos pesos son las entradas
del eigenvector correspondiente al eigenvalor más pequeño. Así podemos
ver que la varianza de las entradas de \texttt{comb\_lin} es \emph{casi}
cero.

\begin{verbatim}
## [1] 7.456453e-19
\end{verbatim}

\begin{Shaded}
\begin{Highlighting}[]
\KeywordTok{var}\NormalTok{(comb_lin)}
\end{Highlighting}
\end{Shaded}

\begin{verbatim}
## [1] 2.336397e-32
\end{verbatim}

c)También se puede calcular la varianza muestral \(S_Y\) a partir de
\(C'S_XC\). Comprobamos esto al observar que la norma euclideana de la
diferencia entre las matrices y es \emph{casi} cero.

\begin{Shaded}
\begin{Highlighting}[]
\KeywordTok{norm}\NormalTok{(aprox }\OperatorTok{-}\StringTok{ }\NormalTok{S_y)}
\end{Highlighting}
\end{Shaded}

\begin{verbatim}
## [1] 3.400058e-16
\end{verbatim}

\hypertarget{problema-6}{%
\section{Problema 6}\label{problema-6}}

Para la matriz de covarianza alrededor de \(x=a\)
\[S(a) = \frac{1}{n}\sum_{i=1}^{n} (x_{i}-a)(x_{i}-a)' \] a) Por
demostrar : \[S(a) = S + (\overline{x} - a )(\overline{x} - a )' \]

Recordemos que el producto exterior \(f(x,y) = xy'\) es una función
bilineal y tenemos que :

\begin{align*}
&(x_{i}-a)(x_{i}-a)' = \\
&(x_{i}-\overline{x}+\overline{x}-a)(x_{i}-\overline{x}+\overline{x}-a)' = \\ &(x_{i}-\overline{x}+\overline{x}-a)(x_{i}-\overline{x})' + (x_{i}-\overline{x}+\overline{x}-a)(\overline{x}-a)' = \\
&(x_{i}-\overline{x})(x_{i}-\overline{x})'+ (\overline{x}-a)(x_{i}-\overline{x})'+ (x_{i}-\overline{x})(\overline{x}-a)'+ (\overline{x}-a)(\overline{x}-a)' = \\
&(x_{i}-\overline{x})(x_{i}-\overline{x})'+ 2(\overline{x}-a)(x_{i}-\overline{x})'+ (\overline{x}-a)(\overline{x}-a)'
\end{align*}

Así \begin{align*}
&S(a) = \frac{1}{n}\sum_{i=1}^{n} (x_{i}-a)(x_{i}-a)' =\\
&\frac{1}{n}\sum_{i=1}^{n}(x_{i}-\overline{x})(x_{i}-\overline{x})'+ \frac{2}{n}\sum_{i=1}^{n}(\overline{x}-a)(x_{i}-\overline{x})'+ \frac{1}{n}\sum_{i=1}^{n}(\overline{x}-a)(\overline{x}-a)' = \\
&S + 2(\overline{x}-a)(\frac{1}{n}\sum_{i=1}^{n}x_{i}-\overline{x})'+(\overline{x}-a)(\overline{x}-a)' = \\
&S + (\overline{x}-a)(\overline{x}-a)'
\end{align*}

\begin{enumerate}
\def\labelenumi{\alph{enumi})}
\setcounter{enumi}{1}
\tightlist
\item
  De a) podemos calcular el determinante de \(S(a)\) recordando que
  \[det(A+ba') = det(A)(1+b'A^{-1}a)\] Así:
\end{enumerate}

\[det(S(a)) = det(S+(\overline{x}-a)(\overline{x}-a)') = det(S)(1+(\overline{x}-a)'S^{-1}(\overline{x}-a))\]

Además, como \(S\) es semidefinida positiva sus eigenvalores son no
negativos y los eigenvalores de \(S^{-1}\) son los recíprocos y
conservan el signo por lo que tambíen es semidefinida positiva y así la
forma cuadrática \((\overline{x}-a)'S^{-1}(\overline{x}-a)\) tiene un
valor mínimo de cero que se alcanza en \(\overline{x} = a\) :

\[min_{a}det(S(a)) = det(S)\]

\end{document}
